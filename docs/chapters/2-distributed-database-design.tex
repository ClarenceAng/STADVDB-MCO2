%%
%% Section 2: Distributed Database Design
%%
\section{Distributed Database Design}

% Using Exercise 5 Distributed Databases as your guide, present the setup of your distributed database.

\subsection{System Architecture}
% Present the setup of your distributed database
% - The data stored in each node
% - The application use case (functionalities)
% Provide illustrations and diagrams as necessary

% TODO: Add system architecture diagram
% \begin{figure}[h]
%   \centering
%   \includegraphics[width=\linewidth]{images/architecture.png}
%   \caption{Three-Node Distributed Database System Architecture}
%   \label{fig:architecture}
% \end{figure}

The distributed database system consists of three nodes deployed on separate machines:

\begin{itemize}
    \item \textbf{Node 1 (Central Node):} Contains all rows in the main database, serving as the master node for data replication and coordination.
    \item \textbf{Node 2:} Contains a horizontal partition of the data based on [specify fragmentation criterion, e.g., year < 2010].
    \item \textbf{Node 3:} Contains the complementary horizontal partition based on [specify fragmentation criterion, e.g., year >= 2010].
\end{itemize}

\subsection{Database Schema}
% Present the database schema used across all nodes
% Note: The schema is homogeneous across all nodes

% TODO: Add database schema description
% \begin{verbatim}
% CREATE TABLE table_name (
%     column1 datatype,
%     column2 datatype,
%     ...
% );
% \end{verbatim}

\subsection{Data Fragmentation Strategy}
% Is the database homogeneous or heterogeneous?
% How is data fragmented across the nodes?

The distributed database follows a \textbf{homogeneous} design where the schema across all three nodes is identical. Data fragmentation is implemented using \textbf{horizontal fragmentation} based on [specify criterion].

% TODO: Explain fragmentation criteria and justify the choice

\subsection{Data Replication Strategy}
% How is data replicated across the nodes?

% TODO: Describe replication strategy
% - Full replication on Node 1
% - Partial replication on Nodes 2 and 3

\subsection{Web Application Overview}
% Describe how the web application manages access to the distributed database

% TODO: Add web application architecture diagram
% \begin{figure}[h]
%   \centering
%   \includegraphics[width=\linewidth]{images/webapp-architecture.png}
%   \caption{Web Application Architecture}
%   \label{fig:webapp}
% \end{figure}

\subsection{Update and Recovery Mechanism Overview}
% Give an overview of the update and recovery mechanism as a prelude to Sections 3 and 4

% TODO: Provide high-level overview
% - How updates are propagated across nodes
% - How recovery is handled when nodes fail