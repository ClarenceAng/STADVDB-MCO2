%%
%% Section 3: Concurrency Control and Consistency
%%
\section{Concurrency Control and Consistency}

% How do you maintain global concurrency control through your web application and update strategy?

\subsection{Update Strategy}
% Describe your update strategy to maintain database consistency
% - Updates in Node 2 or Node 3 should be replicated to the central node
% - Updates in the central node should be replicated to Node 2 or Node 3

% TODO: Describe whether master-slave, multi-master, or other setup
% Justify your choice

\subsubsection{Replication Algorithm}
% Illustrate with the use of an algorithm (not source code)

% TODO: Add algorithm
% \begin{algorithm}
% \caption{Update Replication Algorithm}
% \begin{algorithmic}[1]
% \State // Pseudocode for update replication
% \If{update on Node 2 or Node 3}
%     \State Replicate to Central Node (Node 1)
% \ElsIf{update on Central Node}
%     \State Determine target node based on fragmentation criteria
%     \State Replicate to appropriate node (Node 2 or Node 3)
% \EndIf
% \end{algorithmic}
% \end{algorithm}

\subsubsection{Data Transparency}
% How is data transparency supported by your data replication and update strategy?

% TODO: Explain how users are shielded from the distributed nature of the database

\subsection{Isolation Level Selection}
% Which isolation level is most appropriate for your web application?
% Justify based on experiments

The four isolation levels tested are:
\begin{enumerate}
    \item \textbf{Read Uncommitted:} Allows dirty reads, non-repeatable reads, and phantom reads.
    \item \textbf{Read Committed:} Prevents dirty reads but allows non-repeatable reads and phantom reads.
    \item \textbf{Repeatable Read:} Prevents dirty reads and non-repeatable reads but allows phantom reads.
    \item \textbf{Serializable:} Prevents all anomalies but has the highest overhead.
\end{enumerate}

% TODO: Add table comparing isolation levels and their performance
% \begin{table}[h]
%   \caption{Isolation Level Comparison}
%   \label{tab:isolation}
%   \begin{tabular}{lcccc}
%     \toprule
%     \textbf{Isolation Level} & \textbf{Throughput} & \textbf{Consistency} & \textbf{Anomalies} \\
%     \midrule
%     Read Uncommitted & High & Low & All \\
%     Read Committed & Medium-High & Medium & Some \\
%     Repeatable Read & Medium & High & Few \\
%     Serializable & Low & Highest & None \\
%     \bottomrule
%   \end{tabular}
% \end{table}

\subsection{Methodology}
% Explain the methodology employed to simulate global concurrency control

\subsubsection{Experimental Setup}
% Describe the setup of your experiments

% TODO: Describe hardware, software, network configuration

\subsubsection{Test Cases}
% Show the concurrent transactions used to simulate the experiment/test cases

\paragraph{Case \#1: Concurrent Read Operations}
% Concurrent transactions in two or more nodes are reading the same data item

% TODO: Describe test case setup and transactions

\paragraph{Case \#2: Concurrent Read-Write Operations}
% At least one transaction is writing (update/delete) and other concurrent transactions are reading the same data item

% TODO: Describe test case setup and transactions

\paragraph{Case \#3: Concurrent Write Operations}
% Concurrent transactions in two or more nodes are writing (update/delete) the same data item

% TODO: Describe test case setup and transactions

\subsubsection{Validation Method}
% How did you validate that the concurrency control, replication and update strategies leave the database in a consistent state?

% TODO: Describe validation approach

\subsection{Results and Discussion}
% Show your test results and give your analysis
% Use tables and diagrams accordingly

% TODO: Add results tables and analysis
% \begin{table}[h]
%   \caption{Concurrency Control Test Results}
%   \label{tab:concurrency-results}
%   \begin{tabular}{lccc}
%     \toprule
%     \textbf{Test Case} & \textbf{Trial 1} & \textbf{Trial 2} & \textbf{Trial 3} \\
%     \midrule
%     Case \#1 & Pass/Fail & Pass/Fail & Pass/Fail \\
%     Case \#2 & Pass/Fail & Pass/Fail & Pass/Fail \\
%     Case \#3 & Pass/Fail & Pass/Fail & Pass/Fail \\
%     \bottomrule
%   \end{tabular}
% \end{table}

% Discuss key insights from your replication and update strategies
% Support with relevant literature

% TODO: Analyze results and compare with prior works