%%
%% Section 4: Global Failure and Recovery
%%
\section{Global Failure and Recovery}

\subsection{Recovery Strategy}

\subsubsection{Recovery Operation Process}
% What happens during a recovery operation?
% What facilities did you build to support your distributed database during a recovery operation?

% TODO: Describe recovery process

\subsubsection{Recovery Algorithm}
% Illustrate with the use of an algorithm (not source code)

% TODO: Add recovery algorithm
% \begin{algorithm}
% \caption{Node Recovery Algorithm}
% \begin{algorithmic}[1]
% \State // Pseudocode for node recovery
% \When{node comes back online}
%     \State Check for missed transactions from transaction log
%     \State Apply missed transactions in order
%     \State Synchronize with central node
%     \State Resume normal operations
% \EndWhen
% \end{algorithmic}
% \end{algorithm}

\subsection{Methodology}
% Explain the methodology employed to simulate global failure and recovery

\subsubsection{Experimental Setup}
% Describe the setup of your experiments

% TODO: Describe how failures were simulated (e.g., network interruption, process termination)

\subsubsection{Test Cases}
% Present the test cases used to simulate each of the test cases in global failure and recovery

\paragraph{Case \#1: Central Node Write Failure During Replication from Node 2/3}
% When attempting to replicate the transaction from Node 2 or Node 3 to the central node, 
% the transaction fails in writing (insert/update) to the central node.

% TODO: Describe test case setup and expected behavior

\paragraph{Case \#2: Central Node Recovery with Missed Transactions}
% The central node eventually recovers from failure and missed certain write transactions.

% TODO: Describe test case setup and expected behavior

\paragraph{Case \#3: Node 2/3 Write Failure During Replication from Central Node}
% When attempting to replicate the transaction from central node to Node 2 or Node 3,
% the transaction fails in writing (insert/update) to Node 2 or Node 3.

% TODO: Describe test case setup and expected behavior

\paragraph{Case \#4: Node 2/3 Recovery with Missed Transactions}
% Node 2 or Node 3 eventually recovers from failure and missed certain write transactions.

% TODO: Describe test case setup and expected behavior

\subsubsection{Validation Method}
% How did you validate the correctness of your recovery strategy?

% TODO: Describe validation approach

\subsection{Results and Discussion}
% Show your results and give your analysis
% Use tables and diagrams accordingly

% TODO: Add results tables
% \begin{table}[h]
%   \caption{Global Failure Recovery Test Results}
%   \label{tab:recovery-results}
%   \begin{tabular}{lccc}
%     \toprule
%     \textbf{Test Case} & \textbf{Trial 1} & \textbf{Trial 2} & \textbf{Trial 3} \\
%     \midrule
%     Case \#1 & Pass/Fail & Pass/Fail & Pass/Fail \\
%     Case \#2 & Pass/Fail & Pass/Fail & Pass/Fail \\
%     Case \#3 & Pass/Fail & Pass/Fail & Pass/Fail \\
%     Case \#4 & Pass/Fail & Pass/Fail & Pass/Fail \\
%     \bottomrule
%   \end{tabular}
% \end{table}

% Discuss key insights from your recovery strategies
% Support with relevant literature

\subsubsection{User Shielding from Node Failure}
% How are users shielded from node failure?
% When a node fails, how does your data replication strategy support the continued availability of the web application?

% TODO: Explain user experience during failures

\subsubsection{Recovery Support}
% How does your data replication strategy support the recovery of your distributed database system from failure?

% TODO: Explain how replication aids recovery